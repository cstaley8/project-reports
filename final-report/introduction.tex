\chapter{Introduction}

The CeeNC is a web enabled \gls{cnc} interface designed with students and hobbyists in mind. 
While most modern \gls{cnc} systems are expensive and large, the CeeNC is an affordable and compact device. 
The device can also be configured for multiple setups, like a linear \gls{cnc}, a Delta Robot, or a 3D printer. 
The web interface also improves upon current \gls{cnc} computer interfaces, which are not user friendly, and allows the device to be used from anywhere on the network.

In most cases, owning a \gls{cnc} requires a large monetary investment, along with a sizable area to keep it in. 
This makes it hard for students and hobbyists, along with some companies, to be able to have their own \gls{cnc}.
A personal \gls{cnc} can cut down on project development time.
The user interfaces for \gls{cnc} control that are currently availble are not known to be user friendly.

The purpose of the CeeNC is to create a \gls{cnc} interface, capable of receiving standardized G-code through \gls{tcpip}, processing the G-code, and driving motors and \gls{gpio} according to the G-code. 
G-code is a standard \gls{cnc} programming language, but is more complicated than accepting individual commands to move the motors.

The following will discuss the need for the product, the process used to design solutions, the solutions that were implemented and how they will work, and how the solutions may affect the world.
Lastly, recommendations will be suggested on how the project can move forward.
