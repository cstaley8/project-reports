\section{Hardware Design}
The \gls{ceenc} consists of three major hardware components, the \gls{pi}, the control board, and the driver board.
The \gls{pi} will handle most of the processing and have a master relationship with the control board.
The control board will be an additional microcontroller that supports the \gls{pi} in timing and motor control functions.
The driver board will receive motor control commands from the control board and execute them using the \gls{ti} DRV8825 motor driver.
The driver board was developed and implemented separately from the control board to allow for early and individual testing.

\subsection{Raspberry Pi}
The \gls{pi} will receive G-code from the web interface over \gls{tcpip} through the Ethernet or USB ports.
The \gls{pi} will send commands to the motor control board microcontroller over a \gls{spi} bus.
The \gls{pi} will send data to a port expander on the control board over an \gls{i2c} connection.
The \gls{pi} will operate in the master mode for the \gls{spi} communication.
The control board will be dependent on the information it receives from the \gls{pi} for motor operation.
\gls{spi} was chosen because it is native on the \gls{pi} and most microcontrollers support it.
The \gls{pi} will be powered from the +5V rail on the control board.

\subsection{Control Board}
The control board will be implemented using the TI C2000 microcontroller.
A microcontroller supports the \gls{pi} so that the stepper motor timers can be run independently from the overhead of an operating system.
The microcontroller will receive commands from the \gls{pi} over a \gls{spi} connection.
The control board will contain an \gls{i2c} port expander that will provide 16 general purpose outputs and is controlled from the \gls{pi}.
The control board will send commands to the driver board over a parallel port connection.
A parallel port was chosen so that the driver board would also be able to interface with the parallel port on some PCs or other \gls{cnc} drivers.
The parallel port also allows for the I/O of the microcontroller to be mapped to the I/O of the motor drivers in a 1-to-1 relationship.
The microcontroller will output step, direction, and enable data for motor control.
It will receive fault and home inputs from the motor driver board.
The step lines must be capable of outputting a 10kHz signal.
A step frequency of 10kHz will be sufficient for \gls{cnc} applications.
The microcontroller uses 4 timers to track of the step count of each motor. 
The control board will have one DC power input.
This power input can range from 14V-36V.
There will be two regulated power supplies on the control board, 5V and 3.3V.
The 5V supply will power the \gls{pi}.
The 3.3V supply will power the microcontroller, the \gls{i2c} port expander, and the opto-isolators on the motor driver board.
The 14V-36V supply will power the motor drivers.

\subsection{Driver Board}
The motor drivers will be implemented using four \gls{ti} DRV8825.
The DRV8825 is a dual stepper/DC motor driver.
Each DRV8825 will receive step and direction commands from the motor control board over a parallel port connection.
The motor driver ICs will step the motors according to the commands from the microcontroller.
The motor driver board contains any necessary pull up/pull down resistors for the motor control signals.

\subsection{Recommendations and Actions Taken}
