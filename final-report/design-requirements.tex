\chapter{Design Requirements and Success Criteria}
The project's success will be determined by whether or not the 10 engineering requirements, consisting of 5 \gls{pcsc} and 5 \gls{pssc}, are met. 

\section{Five Project Common Success Criteria}
\begin{enumerate}
	\item Create a complete \gls{bom} and order/sample all parts needed for the design.
	\item Develop complete, accurate, readable schematic of the design, complete with interface loading analysis and interface timing analysis. 
	\item Complete a layout and etch a \gls{pcb}.
	\item Populate and debug the design on a custom \gls{pcb}.
	\item Professionally package the finished product and demonstrate its functionality.
\end{enumerate}

\section{Five Project Specific Success Criteria}
\label{sec:psscs}
\begin{enumerate}
	\item The system will drive at least 4 stepper motors, 1 DC motor, and 16 General Purpose Outputs.
The system will receive inputs from at least 1 emergency stop switch and 4 stepper motor home inputs.
	\item The system will Receive G-code through \gls{tcpip}.
The system’s software will be developed using IEEE Std 830-1998 Recommended Practice for Software Requirements Specifications.
	\item The step frequency range will be at a minimum of 10kHz.
For any chosen frequency in range, the actual step frequency will be within 5\% of the desired frequency.
	\item The system power supply will accept between 14V and 36V and draw a maximum of 10A, including current required for the motors. 
	\item The system will stop all motors and shutdown all microcontrollers if the main microcontroller temperature reaches $60^{\circ}C$.
\end{enumerate}

\section{Design Constraints}
The major design constraints considered in this report are computational requirements, interface requirements, on-chip peripheral requirements, off-chip peripheral requirements, power constraints, packaging constraints, and cost constraints.

\subsection{Computational Requirements}
The main computational requirement for this project is the conversion of G-code commands to motor control functions.
The secondary computational requirement is the count and timing of stepper motor steps.
The \gls{pi} will handle software conversion while the microcontroller will handle the timing and count of the motor steps.
When a G-code design file is uploaded and sent to the \gls{pi}, the \gls{pi} will convert the code into commands that can be sent to the microcontroller over a \gls{spi} bus.
This conversion may occur in non-real-time.
The microcontroller will handle the timing and counting of the motor control functions.
Once the microcontroller receives the motor control commands, it will output appropriate step and direction data to the motor drivers.
The motor control data must execute in real-time to ensure accuracy.
The microcontroller must be capable of outputting a step frequency of at least 10kHz and within 5\% of the target frequency for any frequency in that range.
The microcontroller will also monitor the status of the motor drivers and communicate any faults back to the computer.

\subsection{Interface Requirements}
The microcontroller must have at least 8 outputs for stepper motor control, 4 inputs for stepper motor home, 1 input for the motor driver faults, 1 input for an emergency stop, and 1 PWM output for DC motor control.
Between the computer and microcontroller, an additional 16 outputs must be available for use by the end user as part of the project success criteria.
The additional outputs may be accomplished by using serial to parallel shift registers or an \gls{i2c} port expander.
If shift registers are used, 3 output pins will be required by either the pi or microcontroller for the clock, storage input, and data input.
If an \gls{i2c} port expnader is used, 2 output pins will be required by either the \gls{pi} or the microcontroller for the \gls{i2c} channel.
All digital logic will be 3.3V.
To protect the \gls{pi} and microcontroller, opto-isolators will be used at the input and output of the motor drivers.

\subsection{On-Chip Peripheral Requirements}
The \gls{pi} will require either Ethernet or Wi-Fi capabilities to connect to a website.
The \gls{pi} must have a 1 channel \gls{spi} bus to communicate with the microcontroller. 
The microcontroller must have a 1 channel \gls{spi} bus to communicate with the \gls{pi}.
The microcontroller must have a JTAG port for programming.
The microcontroller must have 4 16-bit timers for motor step counting.

\subsection{Power Constraints}
The system will accept one DC power supply that can be between 14V and 36V.
The system will draw no more than 10A including the current required for the motors.
A 5V power regulator will be used to supply the computer.
A 3.3V power regulator will be used to supply the microcontroller and ICs.
The 14-36V DC rail will be used to supply the motor drivers. 

\subsection{Packaging Constraints}
The finished project must be professionally packaged.
The footprint of the packaging will likely not be much larger than the footprint of largest PCB which is expected to be approximately 87mm x 56 mm, however this is not a requirement.
The size of the finished project will not impact performance in a considerable manner.

\subsection{Cost Constraints}
Current CNC interfaces require use of a full computer system in combination with a motor driver platform.
These setups can cost upwards of \$500.
This project will combine and simplify these hardware and software requirements for a cost of under \$100.