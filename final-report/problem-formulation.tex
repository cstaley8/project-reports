\chapter{Problem Formulation}
\section{Problem Statement}
The modern \gls{cnc} interface is limited to utilization of a full computer system, in combination with a motor driver platform to for complete functional control.
This setup can cost upwards of \$500, depending on the quality and system specifications.
This is not affordable for personal use, so many students and hobbyists
2. Client – who will use the result of this project?

Aimed at hobbyists and students who want an affordable CNC interface.

3. What is the problem to be solved?  Why is this project needed?
This system will encapsulate these hardware and software requirements completely for less than \$100, bringing the \gls{cnc} interface to a price point comparable with that of the modern printer. 
The CeeNC is designed to be an affordable CNC interface that is user friendly and compact.

\section{Background}

State of the Art

1. What are the present relevant solutions?

CNC interfaces that come with the devices or are availble, yet they are unintuative and not user friendly.

2. What products are presently available?
\subsection{Competitive Products}


\textbf{MakerBot}

The MakerBot could be considered the leader in the current 3D printer market.
It offers a build area of 410 cubic inches and a layer resolution of 100 microns, which is about the width of a piece of printer paper. It comes preassembled in an enclosed steel case. 
The work area can be manually leveled for ease of calibration. 
It uses a linear mill design for its work head control, comes with its own software for breaking apart the models, and has an easy to understand user interface. 
The user can send models to be printed by USB or SD card. 
The cost of the MakerBots are around 2000 dollars, depending on the model.

\textbf{DeltaMaker}

The DeltaMaker is designed by a different group, but has similar specifications. 
It uses a Delta Robot to control its work head. The printer has a 100 micron layer resolution and a 10 inch diameter by 11 inch tall work space. 
The frame is an open aluminum based frame that allows the project to be viewed from any side. 
It uses available open source tool chains to send over the data for the models through USB. 
It costs 1,999 dollars and is currently on backorder with 12 to 14 weeks of lead time.  

\textbf{RepRap}

The RepRap is a more open source approach to a 3D printer. 
There are many different types of the RepRap. 
Their website, reprap.org, has the instructions to gather the parts and assemble each type. 
It uses a linear design for controlling its X, Y, and Z axis’s. 
The work area varies from type to type, while many of the designs do not even have an estimated size listed. 
The largest specific size listed for a work area is 9"x10"x7" for the X, Y and Z directions. 
Most of the designs cost anywhere from 500 to 1500 dollars for the parts. 
There is one version of the RepRap that offers multiple colors of filament.

\textbf{Rostock}

The Rostock is another design that is supported by the RepRap. 
It is design using a Delta Robot to control the work heads position. 
Their design page on RepRap’s website lists that the design will cost about 500 dollars for the hardware and that it has a work area of 8x8x16 inches. 
It uses an Arduino to handle the G Code processing that is sent by USB. 
The assembly instructions do refer to their current firmware as “a pretty hacky proof of concept and not a long term solution” [1] though.

\textbf{OtherMill}

STILL NEEDS TO BE WRITTEN


\section{Results of Patent and Product Search}


While there is a patent for a generic CNC [1], this project does not actually handle the mechanical parts of a CNC.
So it does not apply. 
Another point of concern was the use of G-code. G-code was patented by Fanus Ltd [2]. 
The patent was filed on April 1, 1983. 
Any patent filed before 1994 had a term of 17 years, meaning that the patent expired in 2000. 
The discussed patents will focus on methods of interaction with the controller. 
A search for method based patents was done for patents on control equations. 
These seem like they had potential for uniqueness but no patents for control equations on CNC devices were found. 
There were a lot on equations and methods used for navigation calculations [3], but those applied to transportation, specifically airplanes These are the areas of highest potential for liability, due to their similarity to popular commercial products.


\subsection{Numerical control unit with set amount of execution}

\textbf{Publication Number:} US 8036770 B2 \\
\textbf{Filing Date:} April 4th, 2008 \\
\textbf{Condensed Abstract:} \\
This patent is for a machine that uses numerical controls. 
It will start a command or set of commands when its start button is depressed. 
It will also suspend execution if there is a change in direction or a non-cutting command is issued. 
It will then wait for the start button to be depressed again before resuming operation [4]. 

\subsection{Approach For Printing To Web Services-Enabled Printing Devices}

\textbf{Publication Number:} US 20100225958 A1 \\
\textbf{Filing Date:} March 6th, 2009 \\
\textbf{Condensed Abstract:} \\
This patent describes a method for retrieving printer data and displaying what functions it has available on a second party application.  
A print driver will hold all the information and the information can be requested by the web service. 
The web service will then parse this data and show what options the printer has available and allow print jobs to be made. 
Then data and options can be sent back to the printer to create and execute print jobs [5].

\subsection{Control device of electric motor}

\textbf{Publication Number:} US 8598818 B2 \\
\textbf{Filing Date:} July 29th, 2005 \\
\textbf{Condensed Abstract:} \\
This patent describes a method of motor control. 
It breaks the control down into three parts, driving, monitoring, and stopping. 
It focuses on stopping the motor in a safe matter, by using the monitoring to determine if the motor is operating under safe conditions. 
Specifically, it looks at the velocity of a motor to check if it needs to be forcibly stopped when an emergency stop button is pressed [6].

\subsection{SENA’s Products}

A product search did not turn up any network enabled CNC devices.  
It did turn up a series of devices that would allow a company to connecting existing devices to their network though. 
SENA has a series of products that connects the CNC to the network through Ethernet, WiFi or Bluetooth. 
There would be potential for infringement on these devices. 
SENA’s United States office declined to comment on any patents being filed on any of their devices. 
A patent search ran with SENA being the assignee returned no hits on patents related to these products.


\section{Analysis of Patent Liability}


Looking at the previously mentioned patents and products, there is a high chance of patent liability.  
The first patent discussed is especially worrisome. 
A lot of the claims are very similar to what our product will be doing. 
Using standardized G-code to process what the print job is supposed to be is going to work the same. 
There is no way around that. 
The patent for G-code has expired, so nobody holds control over the use of G-code. 
This patent does hold control of how G-code is used though.  
Some of its claims even go as far as to mention specific command signifiers. 
The major differences between our product and the product mentioned in patent one are when their product stops and the types of motors used to drive the CNC. 
While our product does not include the mechanical side of the CNC, it is still designed to drive four stepper motors. 
This patent states that it will use servomotors, as shown in their first figure of the patent [Appedix Figure 1]. 
Secondly, their claims state that any time there is a change in the cutting direction or a non-cutting command is sent, the machine will pause and wait for the start button to be pushed again. 
Our product will run a whole batch of control without stopping or prompting for user input. 
The patent also mentions running in a mode that would run one command at a time, which is not what our product would do.
Overall, it is surprising that this vague system was patentable.


Patent two is not as troubling. 
While their claims do state that they will use a print driver, most devices use drivers. 
Since our device will be configurable, our driver will not return to the host the configurations of the printer.  
It will make available to the user the options that can be used to configure the printer. 
Secondly, their device sent its data to a second party application. 
Our device will send data to the website, with no second party application needed. 
Storing and retrieving data are both part of the claims, but are general practices that are used often.  
The way the data is retrieved, through the second party application, will be where there is the difference. 
The data will be directly sent to and stored on the device in the CNC Interface. 
This also covers the claims that state how options will be passed to the system.


Patent three has some potential for causing issue, but the wording will show some separation. 
While the CNC Interface will monitor the motor, it will monitor the position and not the velocity. 
Also, a safety switch will stop the motors, but the safety switch will not engage the motor monitoring as mentioned in the patent. 
The CNC Interface will be constantly monitoring the motors. 
Lastly, there will be no need to forcibly stop the motors as stated in patent 3. 
The CNC Interface is designed to use stepper motors to move a work head, which will stop promptly.

\section{Actions Recommended}


The first patent is unavoidable.
It covers a large area of CNCs because it has a method of how to use G-code, which itself is a standard. 
Trying to redesign around this patent will cause the project to be redesigned from scratch.  
It is doing substantially the same thing in substantially the same way. 
The difference of the continuous movement versus the start and stop mentioned in the patent is not a large enough difference to have the products be different.  
The major difference is in how the motors are driven.  
There is a major difference between the type of signals needed to drive a stepper motor versus a servo motor. 
In this situation, the difference in software and hardware will be substantial. 
With this information, there will be no liability. 
If there is, it will be able to be argued in court. 
At the very least, a settlement can be reached.


Patent 2 does not show any potential for liability. 
Their product centers around data being sent through the network using a second party application.  
That is substantially different. 
The way the patent will have to work is use a standard communication setup to make sure the data is in a correct format for any second party application. 
The CNC Interface will work using point to point communication.  
Network enabled printers are prevalent in today’s world.


The third patent shows little chance for patent liability also. 
While the motor driving is the same end product, the way it will happen is not the same. 
The monitoring is also different in method and end results.

