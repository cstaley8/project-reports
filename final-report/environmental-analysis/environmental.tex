\chapter{Social, Political, and Environmental Impact}

The \gls{ceenc} was not developed without regarding how the end product would affect the world.
Conscientious effort was put forth in understanding how the \gls{ceenc} would affect its users, how long it would last, and what affects it may have on the environment, especially upon disposal.

\section{Environmental Analysis}
Environmental impact is the area in which the \gls{ceenc} provides quantifiable benefit. 
Current \gls{cnc} controller implementations are large, and generate a great deal of heat. 
The \gls{ceenc} uses modern drivers to greatly increase efficiency and reduce power consumption. 
Computational components have also been reduced, and are significantly better suited for controlling these devices than full fledged PC towers. 

Combination of the factors above reduces the materials required in construction of the \gls{ceenc} by a factor of 10. 
Although limited considerations have been made pertaining to the ROHS certification of the components involved, it is safe to say that in removal of the PC tower from the control loop, it will most certainly prove to be a net positive for the environment. 
Optimization of power consumption has a twofold impact, not only in conservation of electricity, but in minimization of materials required for heat sinking and structure.

Finally, as mentioned in the economic analysis, \gls{cnc} technology will impact supply and distribution chains worldwide. \cite{3dprintsustain}
While mass manufacturing will maintain its place as a method by which goods can be produced cheaply and effectively, the offset in shipping costs for localized manufacture will see less freight shipped overall. \cite{3dprintenvironment}

\section{Sustainability}
The highly modular construction of the \gls{ceenc} will greatly increase its projected lifespan.
Targeted to audiences in the fields of science and engineering, the construction is such that they will be able to repair or re-purpose components as they see fit.
In separation of the computational, control, and driver components, persons are able to integrate any section of the system with existing devices.
For example, the control card is capable of controlling not only the custom driver card, but any number of \gls{cnc}s using the standard interface as well.
The same works in reverse for the driver card, and should the end user see fit, the computational control can be replaced by any system with an \gls{spi} interface.

For projected long term growth of the \gls{ceenc} brand, the technology utilized in its design is expected to remain relevant for another 5 to 10 years.
Many current \gls{cnc} systems on the market today still utilize chipsets devised in the late 80's, like the L298. 
The \gls{ceenc} has made use of modern motor drivers and microcontrollers with high industry backing.
Until entire \gls{cnc} systems on a chip go into manufacture, the \gls{ceenc} will remain an optimal means by which consumers can control their \gls{cnc} devices.
 
\section{Ethical}
Legal concerns pertaining to “warning label placement” are greatly simplified by the limited scope of the \gls{ceenc}. 
Safety concerns will arise largely from the machinery which the device is to drive, and as such are specific to its implementation. 
With no lethal voltages, or hazardous \gls{rf} interference produced by the components, it is then the sole duty of the \gls{ceenc} to power off when commanded, and not start on fire. 
These have both been explicitly outlined by the \gls{pssc}s, and are a standard feature in most electronics of this type.

More pertinent to the societal impact of this project, is the management of source documentation and intellectual property. 
With much of the design drawing from the open source community, it is important that the information generated by the exercise be reciprocated by its participants. 
In this manner, the world can be positively impacted in the creation of new solutions to common problems. 

