\chapter{Engineering Requirements Specification}
The project's success will be determined by whether or not the 10 engineering requirements, consisting of 5 \gls{pcsc} and 5 \gls{pssc}, are met. 

\section{Five Project Common Success Criteria}
\begin{enumerate}
	\item Create a complete \gls{bom} and order/sample all parts needed for the design.
	\item Develop complete, accurate, readable schematic of the design, complete with interface loading analysis and interface timing analysis. 
	\item Complete a layout and etch a \gls{pcb}.
	\item Populate and debug the design on a custom \gls{pcb}.
	\item Professionally package the finished product and demonstrate its functionality.
\end{enumerate}

\section{Five Project Specific Success Criteria}\
\begin{enumerate}
	\item The system will drive at least 4 stepper motors, 1 DC motor, and 16 \gls{gpio}.
The system will receive inputs from at least 1 motor limit/emergency stop switch and 4 stepper motor home inputs.
	\item The system will Receive G-code through \gls{tcpip}.
The system’s software will be developed using IEEE Std 830-1998 Recommended Practice for Software Requirements Specifications.
	\item The step frequency range will be at least 10kHz.
For any chosen frequency in range, the actual step frequency will be within 5\% of the desired frequency.
	\item The system power supply will accept between 14V and 36V and draw a maximum of 10A, including current required for the motors. 
	\item The system will stop all motors and shutdown all microcontrollers if the main microcontroller temperature reaches $60^{\circ}C$.
\end{enumerate}

\section{Requirements Validation}
Table ~\ref{table:validation} shows the relationship between the engineering requirements and the marketing requirements.
For a marketing requirement to be validated, the respective engineering requirements must be verified.

\begin{table}[H]
	\caption{Engineering Requirements vs. Marketing Requirements}
	\label{table:validation}
	\centering
	\begin{tabular}{|r |c |c |c |c |c|} 
		\hline\hline
		&1&2&3&4&5\\
		\hline
		\textbf{Accurate} &  & & X & X & \\
		\hline
		\textbf{Quick} &  & & X & X & \\
		\hline
		\textbf{Web Interface} & & X & & & \\
		\hline
		\textbf{Drive Motors} & X & & X & X & \\
		\hline
		\textbf{Handle \gls{gpio}} & X & & & X & \\
		\hline
		\textbf{Emergency Stop} & X & & & X & \\
		\hline
		\textbf{Thermal Shutdown } & X & & & X & X \\
		\hline
		\textbf{Supply Range} & & & & X & \\
	\hline 
	\end{tabular}
\end{table}
