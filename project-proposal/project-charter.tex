\chapter{Project Charter}
\section{Project Approach}
The purpose of this project is to create a final working system that not only meets, but exceeds, the expectations of all stakeholders. This goal will be achieved through proper planning and execution by talented engineers with a focus on system testing throughout the process.

\subsection{Roles and Authority}
The following definitions will serve as the guidelines for each team member's major tasks throughout the project lifetime. All team members are expected to help one another out to ensure the best outcomes for the project, not just for the individual's task completion
\paragraph{Faculty Sponsor: Herbert Detloff}
Demonstrates previous experience with several successful capstone design projects.
Provides feedback and assessment of project definition, plan, and status.
Ensures that all team members are aware of the engineering impact of the project on the University of Nebraska - Lincoln and the rest of the world.
Teaches standard engineering principals for project management, development, execution, and testing to aid in the project's completion.

\paragraph{Resource Manager: James Gehringer}
Responsible for promoting collaboration, communicating progress with the faculty sponsor, and procuring resources.
Accountable for intellectual property management in solidification of ideas and iterative verification of project objectives.
Involved with all phases of design in order to better assure both long and short term goals are met with efficiency by keeping track of the progress of the project by setting deadlines, scheduling tasks, and monitoring progress through tools such as Gantt charts.
Oversees documentation and monitors project health to identify issues as soon as possible.

Reports to the Faculty Sponsor. 

\paragraph{Systems Engineer: Evan Milton}
In charge of generating the system's requirements and specifications and understanding the project as a whole, including mechanical, electronic, and software components and their interfacing.
Works closely with the Hardware and Software Development Engineers to best meet the team objectives, system's requirements, and specifications.
Responsible for acceptance testing the prototypes and final project to ensure quality and confirm that all specifications are met.
Determines alternate solutions case any solution fails or becomes unfeasible during development.
Makes final design decisions when challenges arise that require major modifications.

Reports to the Resource Manager and Faculty Sponsor.

\paragraph{Hardware Development Engineer: Chad Staley}
Accountable for developing the design, layout, construction, and testing of the electrical systems required for this project to create a solid platform for the software.
Focused on simplicity and robustness in design, to aid troubleshooting when software is added to the project.
Understands and can communicate the software interfacing that must occur to make the hardware function properly.

Reports to the Systems Engineer, Resource Manager, and Faculty Sponsor. 

\paragraph{Software Development Engineer: Josh Dewitt}
Responsible for the higher level functionality of the project by writing clean, modular code that can be adapted where hardware changes might be cost-prohibitive.
Will communicate with the Hardware Development Engineer to understand the hardware interfacing to meet the requirements set by the Systems Engineer.
Performs domain, software element, and requirements analysis to ensure the code produced matches the needs of the users, then develops code in any language necessary, using test­-driven development.

Reports to the Systems Engineer, Resource Manager, and Faculty Sponsor.

\subsection{Quality Objectives}
The team agrees that output of this capstone design sequence will be high-quality product that is reproducible, robust, and goes beyond expectations.
This objective will be achieved by setting engineering requirements that will use all of our talent, while still being achievable given the time frame of the course, and meeting these set requirements through a team effort.
Rapid prototyping will allow issues to be revealed early in the project's execution, making sure that the correct priorities are set to focus on those items that may cause an engineering requirement to not be met.

\section{Risk Management}
Certain risks will be faced throughout the project, though proper avoidance and mitigation plans will allow project objectives to be completed on time.
Risks will be obsessed using AHP and pairwise comparisons.
If possible, quantitative measurements, in terms of financial or class grade loss if the risk is realized, will allow comparison to determine which risks should be focused on to ensure that they are not realized.

\subsection{Risk Identification}
From previous experience, the major risks that will occur in the project will likely be related to increasing scope, unfeasible designs, insufficient resources, delayed shipments, and lack of motivation.

\subsection{Risk Avoidance}
\begin{itemize} \parskip2pt
	\item To avoid increasing scope late in the project, there will be a change moratorium to the project’s scope at the end of the senior thesis proposal class. Any additional substantial features must be put on hold until the end of the semester if time permits; all other requirements must first be completed.
	\item To avoid issues from a design becoming unfeasible, viable alternative designs will be created in the planning phase.
	\item To avoid running short of resources, proper resource management and thorough time and money planning will ensure that all resource expenditure is within budget.
	\item To avoid being affected by delayed shipments, parts will be ordered as soon as they are confirmed to be required.
	\item To avoid slipping motivation, the team will be kept in contact to create an atmosphere of team cooperation and ensure that all team members feel useful in the project.
\end{itemize}	

\subsection{Risk Mitigation}
\begin{itemize}
	\item In the case of required increased project scope, the team will collaborate and decide the new priority of tasks using AHP. The tasks will be worked in the order decided, and the team will understand which items may not be completed in time.
	\item In case a design becomes unfeasible, one of the alternate designs created in the planning phase will be pursued.
	\item In case of insufficient financial resources, the investors will be contacted, or more investors will be located to sponsor the project.
	\item If shipments are delayed, alternative distributors will be researched and ordered from to obtain the part in time.
	\item If there is a lack of motivation on the team, team outings will be planned that do not involve the project, but will allow team bonding and any personal issues to be discussed.
\end{itemize}

\section{Project Control}
The project will focus on the punctual production of functional deliverables at set deadlines, tracked in a Gantt chart.
The Resource Manager will be able to concretely monitor project status, and inform members of changes in course as they become necessary.
Detailed documentation of all efforts will be maintained in each person's engineering logbook.

\subsection{Communication Channel}
Face-to-face group status meetings in PKI 330G will be held to update the status of each deliverable being worked on and provide proactive feedback to keep the project on track.
As much communication as possible should be done in person, as opposed to through phone, text, or email conversations because of the higher information bandwidth.
A wiki for the project, hosted on wikispaces.com, will be used to maintain documentation that will be useful during development and for the final users. 
The website kanbanpad.com, an online project tracking tool, will be used to update the status of tasks between the planning, designing, development, testing, and release phases of project development.
Room PKI 330G will be used for any additional meetings required, which may or may not include the entire team.

\subsection{Communication Frequency}
The face-to-face status meetings will be held twice a week to keep each member aware of the overall progress.
All team members are encouraged to use PKI 330G as the primary work area for the project to increase collaboration and encourage spontaneous discussion regarding the project.
All team members should update the status of their tasks on kanbanpad.com as soon as they change between one of the stages, allowing all members to see the most up-do-date project status wherever they are.