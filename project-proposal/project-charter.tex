\chapter{Project Charter}

\section{Roles and Authority}
The following definitions will serve as the guidelines for each team member's major tasks throughout the project lifetime. 
All team members are expected to help one another out to ensure the best outcomes for the project, not just for the individual's task completion.

\subsection{Faculty Sponsor: Herbert Detloff}
Demonstrates previous experience with several successful Senior Thesis Projects.
Provides feedback and assessment of project definition, plan, and status.
Ensures that all team members are aware of the engineering impact of the project on the University of Nebraska - Lincoln and the rest of the world.
Teaches standard engineering principals for project management, development, execution, and testing to aid in the project's completion.

Reports to the \gls{ceen} Department.

\subsection{Resource Manager: James Gehringer}
Responsible for promoting collaboration, communicating progress with the faculty sponsor, and procuring resources.
Accountable for intellectual property management in solidification of ideas and iterative verification of project objectives.
Involved with all phases of design in order to better assure both long and short term goals are met with efficiency by keeping track of the progress of the project by setting deadlines, scheduling tasks, and monitoring progress through tools such as Gantt charts.
Oversees documentation and monitors project health to identify issues as soon as possible.

Reports to the Faculty Sponsor. 

\subsection{Systems Engineer: Evan Milton}
In charge of generating the system's requirements and specifications and understanding the project as a whole, including mechanical, electronic, and software components and their interfacing.
Works closely with the Hardware and Software Development Engineers to best meet the team objectives, system's requirements, and specifications.
Responsible for acceptance testing the prototypes and final project to ensure quality and confirm that all specifications are met.
Determines alternate solutions in case any solution fails or becomes unfeasible during development.
Makes final design decisions when challenges arise that require major modifications.

Reports to the Resource Manager and Faculty Sponsor.

\subsection{Hardware Development Engineer: Chad Staley}
Accountable for developing the design, layout, construction, and testing of the electrical systems required for this project to create a solid platform for the software.
Will focus on simplicity and robustness in design to aid in troubleshooting when software is added to the project.
Understands and can communicate the software interfacing that must occur to make the hardware function properly.

Reports to the Systems Engineer, Resource Manager, and Faculty Sponsor. 

\subsection{Software Development Engineer: Josh Dewitt}
Responsible for the higher level functionality of the project by writing clean, modular code that can be adapted where hardware changes might be cost-prohibitive.
Will communicate with the Hardware Development Engineer to understand the hardware interfacing to meet the requirements set by the Systems Engineer.
Performs domain, software element, and requirements analysis to ensure the code produced matches the needs of the users, then develops code in any language necessary, using test­-driven development.

Reports to the Systems Engineer, Resource Manager, and Faculty Sponsor.

\section{Development Methodology}
The project definition was chosen to use all of the engineers' talent, while still being achievable given the time frame of the course.
The goal of this project is to create a final high-quality system that is reproducible, robust, and goes beyond expectations.
This goal will be achieved through proper planning, developing hardware and software in parallel, applying rapid prototyping and \gls{xp} techniques to hardware and software development, and focusing on system testing throughout the process.
Development methodologies will determine how quickly and accurately the project plan can be executed. 

\subsection{Hardware}
The goal of the hardware development is to create reliable circuits that meet design requirements.
Circuits will be simulated using CircuitLab when possible and prototyped using a breadboard.
Upon passing initial simulation and testing, hardware prototypes will be quickly developed using a \gls{cnc} to mill \gls{pcb}s developed in Eagle \gls{cad}.
The \gls{pcb}s will then be populated, allowing software integration and testing to begin.
Prototypes that pass testing will be sent to a manufacturer to create a final product.
An accurate \gls{bom} will be generated for all final hardware modules.

\subsection{Software}
The goal of the software development is to create reliable programs that can be understood and maintained over the course of the Senior Thesis Project and beyond.
\gls{xp} techniques will be used for software development to allow greater flexibility in the system and allow more rapid feature implementation and ensure that the software components are fully integrated with the hardware components. 
\gls{xp} techniques will save time in the long run, as undocumented, untested, and difficult-to-understand code will take more time for modifications and consume more resources for debugging issues.
The most notable \gls{xp} techniques to be used are unit testing, pair programming, and stand-up meetings.

\paragraph{Unit Testing}
Programming through unit testing simply means that code for runtime is never written until a failing test proves that it is needed, ensuring that all code is written for a reason.
Invalid inputs can be provided to code to test boundary conditions that are difficult to produce in acceptance testing, decreasing the number of bugs in code.
Unit testing also enforces that the inputs, outputs, and use of code are fully documented, while verifying code throughout development. 

Automated unit testing that can easily be run at any time ensures that code is written modularly, provides verification throughout development, and checks that new changes do not break existing features of the code.
Whenever a test fails after making a change, the developer can analyze the test to understand how their change caused the test to fail, ultimately correcting the problem before it is fully implemented in the system.
 
\paragraph{Pair Programming}
One person on a small project team performing all development, creates "knowledge silos" that can halt development if the person leaves the team.
Software limitations may restrict other design choices, including hardware design, so software knowledge will be spread throughout the team, improving the quality of the final product. 

Pair programming helps prevent small bugs from being introduced into the code, which can save time throughout software development.
Software bugs are most easily found and corrected shortly after being written, as the engineer is still familiar with that code, so the extra hours can quickly pay off.

\paragraph{Stand-up Meetings}
The most important implication of stand-up meetings is the ability to keep software development on time.
The meeting is performed while standing, to ensure brevity and allow all parties to get back to work as soon as possible.
Despite not taking long, simply discussing the current stage in the development process allows individuals to express any potential roadblocks that may exist and discuss issues they are having.
Another individual on the team will often have useful input and may help avoid roadblocks and solve errors, ultimately keeping development on schedule. 

\section{Project Control}
The project will focus on the punctual production of functional deliverables at set deadlines, tracked in a Gantt chart.
The Resource Manager will be able to concretely monitor project status, and inform members of changes in course as they become necessary.
Detailed documentation of all efforts will be maintained in each person's engineering logbook.

\subsection{Communication Channel}
Face-to-face group status meetings in PKI 330G will be held to update the status of each deliverable being worked on and provide proactive feedback to keep the project on track.
As much communication as possible will be done in person, as opposed to through phone, text, or email conversations because of the higher information bandwidth.
A wiki for the project, hosted on wikispaces.com, will be used to maintain documentation that will be useful during development and for the final users. 
The website kanbanpad.com, an online project tracking tool, will be used to update the status of tasks between the planning, designing, development, testing, and release phases of project development.
Room PKI 330G will be used for any additional meetings required, which may or may not include the entire team.

\subsection{Communication Frequency}
The face-to-face status meetings will be held twice a week to keep each member aware of the overall progress.
All team members are encouraged to use PKI 330G as the primary work area for the project to increase collaboration and encourage spontaneous discussion regarding the project.
All team members will update the status of their tasks on kanbanpad.com as soon as they change between one of the stages, allowing all members to see the most up-do-date project status wherever they are.

\begin{minipage}{\textwidth}
\section{Team Member Sign-off}
By signing below, I agree to the project charter and acknowledge my required contributions to the Senior Thesis Project.
I understand that failure to adhere by the rules set forth in this project charter will result in negative consequences as deemed fit by the remaining team members and the Faculty Sponsor. 

\signatures
\end{minipage}
