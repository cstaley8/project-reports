\chapter{Project Charter}
\section{Project Approach}
The purpose of this project is to create a final working system that exceeds the expectations of all stakeholders.
This goal will be achieved through proper planning and execution with a focus on system testing throughout the process.

\subsection{Roles and Authority}
The following definitions will serve as the guidelines for each team member's major tasks throughout the project lifetime. 
All team members are expected to help one another out to ensure the best outcomes for the project, not just for the individual's task completion.
\paragraph{Faculty Sponsor: Herbert Detloff}
Demonstrates previous experience with several successful capstone design projects.
Provides feedback and assessment of project definition, plan, and status.
Ensures that all team members are aware of the engineering impact of the project on the University of Nebraska - Lincoln and the rest of the world.
Teaches standard engineering principals for project management, development, execution, and testing to aid in the project's completion.

Reports to the \gls{ceen} Department.

\paragraph{Resource Manager: James Gehringer}
Responsible for promoting collaboration, communicating progress with the faculty sponsor, and procuring resources.
Accountable for intellectual property management in solidification of ideas and iterative verification of project objectives.
Involved with all phases of design in order to better assure both long and short term goals are met with efficiency by keeping track of the progress of the project by setting deadlines, scheduling tasks, and monitoring progress through tools such as Gantt charts.
Oversees documentation and monitors project health to identify issues as soon as possible.

Reports to the Faculty Sponsor. 

\paragraph{Systems Engineer: Evan Milton}
In charge of generating the system's requirements and specifications and understanding the project as a whole, including mechanical, electronic, and software components and their interfacing.
Works closely with the Hardware and Software Development Engineers to best meet the team objectives, system's requirements, and specifications.
Responsible for acceptance testing the prototypes and final project to ensure quality and confirm that all specifications are met.
Determines alternate solutions case any solution fails or becomes unfeasible during development.
Makes final design decisions when challenges arise that require major modifications.

Reports to the Resource Manager and Faculty Sponsor.

\paragraph{Hardware Development Engineer: Chad Staley}
Accountable for developing the design, layout, construction, and testing of the electrical systems required for this project to create a solid platform for the software.
Will focus on simplicity and robustness in design to aid in troubleshooting when software is added to the project.
Understands and can communicate the software interfacing that must occur to make the hardware function properly.

Reports to the Systems Engineer, Resource Manager, and Faculty Sponsor. 

\paragraph{Software Development Engineer: Josh Dewitt}
Responsible for the higher level functionality of the project by writing clean, modular code that can be adapted where hardware changes might be cost-prohibitive.
Will communicate with the Hardware Development Engineer to understand the hardware interfacing to meet the requirements set by the Systems Engineer.
Performs domain, software element, and requirements analysis to ensure the code produced matches the needs of the users, then develops code in any language necessary, using test­-driven development.

Reports to the Systems Engineer, Resource Manager, and Faculty Sponsor.

\subsection{Quality Objectives}
The team agrees that output of this capstone design sequence will be high-quality product that is reproducible, robust, and goes beyond expectations.
This objective will be achieved by setting engineering requirements that will use all of our talent, while still being achievable given the time frame of the course, and meeting these set requirements through a team effort.
Rapid prototyping will allow issues to be revealed early in the project's execution, making sure that the correct priorities are set to focus on those items that may cause an engineering requirement to not be met.

\section{Project Control}
The project will focus on the punctual production of functional deliverables at set deadlines, tracked in a Gantt chart.
The Resource Manager will be able to concretely monitor project status, and inform members of changes in course as they become necessary.
Detailed documentation of all efforts will be maintained in each person's engineering logbook.

\subsection{Communication Channel}
Face-to-face group status meetings in PKI 330G will be held to update the status of each deliverable being worked on and provide proactive feedback to keep the project on track.
As much communication as possible will be done in person, as opposed to through phone, text, or email conversations because of the higher information bandwidth.
A wiki for the project, hosted on wikispaces.com, will be used to maintain documentation that will be useful during development and for the final users. 
The website kanbanpad.com, an online project tracking tool, will be used to update the status of tasks between the planning, designing, development, testing, and release phases of project development.
Room PKI 330G will be used for any additional meetings required, which may or may not include the entire team.

\subsection{Communication Frequency}
The face-to-face status meetings will be held twice a week to keep each member aware of the overall progress.
All team members are encouraged to use PKI 330G as the primary work area for the project to increase collaboration and encourage spontaneous discussion regarding the project.
All team members will update the status of their tasks on kanbanpad.com as soon as they change between one of the stages, allowing all members to see the most up-do-date project status wherever they are.

\begin{minipage}{\textwidth}
\section{Team Member Sign-off}
By signing below, I agree to the project charter and acknowledge my required contributions to the Capstone Design Sequence. I understand that failure to adhere by the rules set forth in this project charter will result in negative consequences as deemed fit by the remaining team members and the Faculty Sponsor. 

\signatures
\end{minipage}
