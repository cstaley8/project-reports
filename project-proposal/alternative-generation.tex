\chapter{Design Alternatives Generation}
The Level One design criterion are as follows. 

\section{Web Interface}
There are several communication protocols over the internet, all of which are well-supported and have implementations available for use. 
The decision comes down to how secure and easy to use the system needs to be.

\paragraph{Plaintext Socket} 
The Plaintext socket offers an interface that is simple to implement, and easy to debug.
The protocol however requires a higher skill level from its users, and poses security issues.

\paragraph{SSL Socket} 
The SSL socket is a secure method of information transfer utilizing authentication and privacy protection.
This comes with greater development costs, and a higher skill level required from its users. 

\paragraph{HTTPS} 
HTTPS is a modern secure interface that is easy for users to understand and utilize.
This protocol will require a domain name however in order to be implemented.

\paragraph{HTTP}
HTTP is a simple to implement, easy to debug, and low skill level protocol for information transfer.
This comes at a cost of weakened security in its implementation.

\paragraph{SFTP}
SFTP is a secure file transfer protocol.
This method is difficult for users to implement.

\section{Master Controller}
The Master Controller  was selected based upon not only processor power, but community support as well. 

\paragraph{Rasberry Pi}
The \gls{pi} is a single board computer developed by the Rasberry Pi Foundation (RPF) to promote computer science education in schools worldwide.
It is based off of the Broadcom BCM2835 System on a Chip (SOC) featuring a 700 MHz ARM processor, VideoCore IV GPU, and 512 MB of Random Access Memory (RAM).
This enables the Pi to maintain a complete Linux Operating System (OS), at a price point just under 35 US Dollars (USD).
Of major benefit to the Pi is the community that has emerged to support its development post-launch.

\paragraph{BeagleBoard}
The BeagleBoard (BB) is a single board computer developed by Texas Instruments (TI) for the Digi-Key and Newark consumer markets.
It is based off of the OMAP3530 SOC featuring a 600 MHz ARM processor, TMS320C64+ DSP, and 256 MB of RAM.
This enables the BeagleBoard to operate a variety of Linux OS’s, for a cost of 45 USD.
Of major benifit to the BB is its well documented and supported development cycle, as provided by TI.

\paragraph{Arduino Mega}
The Arduino platform is a grouping of microcontroller implementations that utilize the AVR \& ARM chip archtectures, in combination with a custom compiler.
The most powerful device in this series is the Arduino Mega, with a 16 MHz clock speed has 128 KB of meory, and 54 digital GPIO pins.
While not capable of running an entire OS, this microcontroller offers direct register access and is ethernet capable.
The Arduino community is also one of the most active open source hardware communities alive on the web today.


\section{Motor Driver Controller Microcontroller}
There are four microcontrollers that were evaluated for use in the Motor Driver Controller board, the MSP430, the ATMega324P, the AT90USB1287, and the ARM Stellaris.
These microcontrollers were evaluated based on the number of timers they had, their interrupt capabilities, their SPI capabilities and the number of GPIO pins.

\paragraph{MSP430} The MSP430 has 5 timers, which is more than what the project will need.
It does have interrupt capabilities.
The MSP430 does handle SPI communication. 
There are up to 90 GPIO pins.

\paragraph{ATMega324P} The ATMega324P has 3 timers.
There are interrupt capabilities with this microcontroller.
A master/slave SPI serial interface is available.
There are 32 GPIO pins.   

\paragraph{AT90USB1287} The AT90USB1287 has 4 timers, which is what the project needs.
There are interrupt capabilities with this microcontroller.
There are two SPI ports available. 
There are 48 GPIO pins.

\paragraph{ARM Stellaris} The ARM Stellaris has 3 timers.
There are interrupt capabilities with this microcontroller.
It has one SPI port available.
There are up to 36 GPIO pins.

\section{Motor Drivers}
Motor driver selection was based upon basic I/O functionality and power management.

\paragraph{DRV8825}
The DRV8825 is a microstepping bipolar stepper motor driver manufactured by TI.
It features adjustable current limiting, overcurrent and overtemperature protection, and six microstep resolutions (down to 1/32-step).
It operates from 8.2 45 V and can deliver up to approximately 1.5 A per phase without a heat sink or forced air flow (rated for up to 2.2 A per coil with suffcient additional cooling).

\paragraph{A4988}
In the same vein as the DRV8825, the A4988 is a microstepping bipolar stepper motor driver manufactured by Allegro.
The driver features adjustable current limiting, overcurrent and overtemperature protection, and five different microstep resolutions (down to 1/16-step).
It operates from 8 35 V and can deliver up to approximately 1 A per phase without a heat sink or forced air flow (it is rated for 2 A per coil with suffcient additional cooling).

\paragraph{TB6560}
The TB6560 is a Pulse Width Modulation (PWM) chopper-type stepping motor driver IC designed for sinusoidal-input microstep control of bipolar stepping motors.
The TB6560 can be used in applications that require 2-phase, 1-2-phase, 2W1-2-phase and 4W1-2-phase excitation modes.
The TB6560 is capable of low-vibration, high-performance forward and reverse driving of a two-phase bipolar stepping motor using only a clock signal.