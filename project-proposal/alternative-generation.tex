\chapter{Design Alternatives Generation}
A prior goal of this project was to deliver a Delta style \gls{3d} printer.
While this would have delivered a \gls{3d} printer, calculating the movement of the work head and applying it to the acceleration of the motors proved to be a large undertaking that would have been difficult to accomplish in the time allotted.
The current focus of a \gls{cnc} driver will instead be a starting point for any \gls{3d} design including Delta and Cartesian \gls{3d} printers.

Other design alternatives that were considered were the choices for the central processor and motor drivers.
Ultimately, the \gls{pi} was chosen as the processor but another consideration was the BeagleBone Black.
While the BeagleBone Black boasts faster processing speed and more outputs, the \gls{pi} was chosen for its lower cost and available support.
The motor controller chosen was the \gls{ti} DRV8825 over the alternative Allegro A4988. The \gls{ti} DRV8825 has greater step resolution and accepts a larger range of input voltage.

Due to the possibility of timing and overhead issues while timing the movement of 4 stepper motors it was decided that a microcontroller would be used to handle the motor control.
The microcontroller that will be used for this project remains to be determined, but considerations include the TI MSP430 and various Atmel microcontrollers.

\section{Web Interface}
There are four approaches to the interface, using PHP, Javascript, HTML5 or building a stand alone application.
They will be evaluated based on their ability to create a GUI, the current team knowledge and the community knowledge base available. 

\paragraph{PHP} PHP gives a fine selection of tools to create a GUI.
The team also has a solid foundation in PHP development.
PHP also has a large community for development and support.

\paragraph{Javascript} Javascript is well known for its GUI capabilities, offering many different libraries and functions to create dynamic interfaces. 
The team is also knowledgeable on development through Javascript.
The Javascript community is very active and offers a lot of support.

\paragraph{HTML5} HTML5 has the ability to create dynamic interfaces.
The team does have background in HTML5.
The HTML5 documentation and support is not as strong as PHP or Javascript's communities are.

\paragraph{Stand Alone Application} A stand alone application will give complete control over a GUI.
The team does not have much knowledge of how to build a stand alone application, compared to writing in PHP, Javascript, or HTML5.
There will not be much support when building a stand alone application.

\section{Master Controller}
\paragraph{Rasberry Pi}
The \gls{pi} is a single board computer developed by the Rasberry Pi Foundation (RPF) to promote computer science education in schools worldwide.
It is based off of the Broadcom BCM2835 System on a Chip (SOC) featuring a 700 MHz ARM processor, VideoCore IV GPU, and 512 MB of Random Access Memory (RAM).
This enables the Pi to maintain a complete Linux Operating System (OS), at a price point just under 35 US Dollars (USD).
Of major benifit to the Pi is the community that has emerged to support its development post-launch.

\paragraph{BeagleBoard}
The BeagleBoard (BB) is a single board computer developed by Texas Instruments (TI) for the Digi-Key and Newark consumer markets.
It is based off of the OMAP3530 SOC featuring a 600 MHz ARM processor, TMS320C64+ DSP, and 256 MB of RAM.
This enables the BeagleBoard to operate a variety of Linux OS’s, for a cost of 45 USD.
Of major benifit to the BB is its well documented and supported development cycle, as provided by TI.

\paragraph{Arduino Mega}
The Arduino platform is a grouping of microcontroller implementations that utilize the AVR \& ARM chip archtectures, in combination with a custom compiler.
The most powerful device in this series is the Arduino Mega, with a 16 MHz clock speed has 128 KB of meory, and 54 digital GPIO pins.
While not capable of running an entire OS, this microcontroller offers direct register access and is ethernet capable.
The Arduino community is also one of the most active open source hardware communities alive on the web today.


\section{Motor Driver Controller Microcontroller}
There are four microcontrollers that were evaluated for use in the Motor Driver Controller board, the MSP430, the ATMega324P, the AT90USB1287, and the ARM Stellaris.
These microcontrollers were evaluated based on the number of timers they had, their interrupt capabilities, their SPI capabilities and the number of GPIO pins.

\paragraph{MSP430} The MSP430 has 5 timers, which is more than what the project will need.
It does have interrupt capabilities.
The MSP430 does handle SPI communication. 
There is up to 90 GPIO pins.

\paragraph{ATMega324P} The ATMega324P has 3 timers.
There are interrupt capabilities with this microcontroller.
A master/slave SPI serial interface is available.
There are 32 GPIO pins.   

\paragraph{AT90USB1287} The AT90USB1287 has 4 timers, which is what the project needs.
There are interrupt capabilities with this microcontroller.
There are two SPI ports available. 
There are 48 GPIO pins.

\paragraph{ARM Stellaris} The ARM Stellaris has 3 timers.
There are interrupt capabilities with this microcontroller.
It has one SPI port available.
There are up to 36 GPIO pins.

\section{Motor Drivers}
\paragraph{DRV8825}
The DRV8825 is a microstepping bipolar stepper motor driver manufactured by TI.
It features adjustable current limiting, overcurrent and overtemperature protection, and six microstep resolutions (down to 1/32-step).
It operates from 8.2 45 V and can deliver up to approximately 1.5 A per phase without a heat sink or forced air flow (rated for up to 2.2 A per coil with suffcient additional cooling).

\paragraph{A4988}
In the same vein as the DRV8825, the A4988 is a microstepping bipolar stepper motor driver manufactured by Allegro.
The driver features adjustable current limiting, overcurrent and overtemperature protection, and five different microstep resolutions (down to 1/16-step).
It oper- ates from 8 35 V and can deliver up to approximately 1 A per phase without a heat sink or forced air flow (it is rated for 2 A per coil with suffcient additional cooling).

\paragraph{TB6560}
The TB6560 is a Pulse Width Modulation (PWM) chopper-type stepping motor driver IC designed for sinusoidal-input microstep control of bipolar stepping motors.
The TB6560 can be used in applications that require 2-phase, 1-2-phase, 2W1-2-phase and 4W1-2-phase excitation modes.
The TB6560 is capable of low-vibration, high-performance forward and reverse driving of a two-phase bipolar stepping motor using only a clock signal.