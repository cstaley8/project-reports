\chapter{Design Alternatives Generation}
A prior goal of this project was to deliver a Delta style \gls{3d} printer.
While this would have delivered a \gls{3d} printer, calculating the movement of the work head and applying it to the acceleration of the motors proved to be a large undertaking that would have been difficult to accomplish in the time allotted.
The current focus of a \gls{cnc} driver will instead be a starting point for any \gls{3d} design including Delta and Cartesian \gls{3d} printers.

Other design alternatives that were considered were the choices for the central processor and motor drivers.
Ultimately, the \gls{pi} was chosen as the processor but another consideration was the BeagleBone Black.
While the BeagleBone Black boasts faster processing speed and more outputs, the \gls{pi} was chosen for its lower cost and available support.
The motor controller chosen was the \gls{ti} DRV8825 over the alternative Allegro A4988. The \gls{ti} DRV8825 has greater step resolution and accepts a larger range of input voltage.

Due to the possibility of timing and overhead issues while timing the movement of 4 stepper motors it was decided that a microcontroller would be used to handle the motor control.
The microcontroller that will be used for this project remains to be determined, but considerations include the TI MSP430 and various Atmel microcontrollers.